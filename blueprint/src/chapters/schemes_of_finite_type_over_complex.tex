%%% Local Variables: ***
%%% mode:latex ***
%%% TeX-master: "../content.tex"  ***
%%% End: ***


\chapter{Schemes of finite type over $\complex$}


\section{Basic Definitions}

We introduce the notion of schemes of finite type over $\complex$ first, we require the definition to be constructive similar
to {\tt AlgebraicGeometry.Scheme.OpenCover}\uses{AlgebraicGeometry.Scheme.OpenCover}.

\begin{definition}[Schemes locally of finite type over $\complex$]
  \label{def:SchemeLocallyOfFiniteType}
  % \lean{SchemeOfLocallyOfFiniteType}
  A scheme $\schemeOf{X}$ is locally of finite type over $\complex$ is a scheme over $\complex$ togather
  with the following data
    \begin{enumerate}
        \item indexing set: $I$;
        \item a family of finitely generated algebras: $R : I \to \fgalgebra{\complex}$;
        \item a family of open immersions: for each $i \in I$, $\iota_i: \spec{R_i} \to \schemeOf{X}$;
        \item covering: $c : X \to I$ such that for each $x \in X$, $c_x \in \range{\iota_i}$.
    \end{enumerate}
\end{definition}


\begin{definition}[Scheme of finite type over $\complex$]
  \label{def:SchemeOfFiniteType}
  % \lean{SchemeOfFiniteType}
  % \uses{SchemeLocallyOfFiniteType}
  A scheme $\schemeOf{X}$ is of finite type over $\complex$ if it is a scheme locally of finite type over $\complex$
  and has finite indexing set; or equivalently $X$ is quasi-compact.
\end{definition}

\begin{remark}
  If we are only considering schems (locally of) finite type over $\complex$, any morphism of ringed space over $\complex$ is automatically a morphism of {\em locally} ringed space over $\complex$.
\end{remark}


\begin{proposition}
  Let $\schemeOf{X}$ be a scheme locally of finite type over $\complex$, let $U \subseteq X$ be an open subset, then $(U, \mathcal{O}_{X}\mid_{U})$ is a scheme locally of finite type over $\complex$.
\end{proposition}
\begin{proof}
  Since $\schemeOf{X}$ is locally of finite type over $\complex$, it admits an affine open cover $U_{i}\cong\spec{R_{i}}$.
  Then $U = \bigcup_{i} U \cap U_{i}$; since for each $i$, $U \cap U_{i}$ is open in $U_{i} \cong \spec{R_{i}}$, it is a union of $\spec{\left(R_{i}\right)_{f_{j}}}$. This gives us an open covering by finitely generated $\complex$-algebra.

\end{proof}

\begin{proposition}
  Let $\spec A$ be an affine finite scheme over $\complex$. We have that the set of closed points are in bijection to
  $\Hom_{\algebra{\complex}}\left(A, \complex\right)$
\end{proposition}

\begin{proof}
  From~\cref{cor:maximal-ideal-algebra-hom}, we know that for each closed point $\mathfrak m$, i.e. a maximal ideal, there is a unique $\phi_{\mathfrak m} : A \to \complex$ whose kernel is $\mathfrak m$. Conversely, for any $\phi: A \to \complex$, $\ker \phi$ is certainly a prime ideal\footnote{$\ker \phi$ is equal to $(\specop \phi)(\star)$ where $\star$ is the unique point of $\specop\complex$}. Since $\phi$ is surjective\footnote{for each $c \in \complex$, $\phi(c \cdot 1) = c$}, its kernel is maximal.

  It remains to show that $\mathfrak m \mapsto \phi_{\mathfrak{m}}$ and $\phi \mapsto \ker \phi$ are inverse to each other. But this follows from the uniqueness from~\cref{cor:maximal-ideal-algebra-hom}.
\end{proof}


\begin{proposition}
  If $\schemeOf{X}$ is a scheme locally of finite type over $\complex$, then the set of closed points of $X$ is in bijection with $X(\complex) := \Hom_{\sch/\complex}\left(\spec{\complex}, \schemeOf{X}\right)$.
\end{proposition}
