\chapter{Schemes of finite type over $\complex$}

\section{Preliminary Definitions}

Before we can define our main object of interest, {\em schemes of finite type over $\complex$}, we need to introduce some preliminary notions.

\begin{definition}[Morphisms of scheme]
Let $\schemeOf{X}$ and $\schemeOf{Y}$ be two schemes and $\Phi : \schemeOf{X} \to \schemeOf{Y}$ be a morphism of schemes.
\end{definition}

\section{Basic Definitions}

We introduce the notion of schemes of finite type over $\complex$ first, we require the definition to be constructive similar
to {\tt AlgebraicGeometry.Scheme.OpenCover}\uses{AlgebraicGeometry.Scheme.OpenCover}.

\begin{definition}[Schemes locally of finite type over $\complex$]\label{def:SchemeLocallyOfFiniteType}
  % \lean{SchemeLocallyOfFiniteType}
  A scheme $\schemeOf{X}$ is locally of finite type over $\complex$ is a scheme over $\complex$ together
  with the following data
    \begin{enumerate}
        \item indexing set: $I$;
        \item a family of finitely generated algebras: $R : I \to \fgcalgebra{\complex}$;
        \item a family of open immersions: for each $i \in I$, $\iota_i: \spec{R_i} \to \schemeOf{X}$;
        \item covering: $c : X \to I$ such that for each $x \in X$, $c_x \in \range{\iota_i}$.
    \end{enumerate}
\end{definition}


\begin{definition}[Scheme of finite type over $\complex$]\label{def:SchemeOfFiniteType}
  % \lean{SchemeOfFiniteType}
  % \uses{SchemeLocallyOfFiniteType}
  A scheme $\schemeOf{X}$ is of finite type over $\complex$ if it is a scheme locally of finite type over $\complex$
  and has finite indexing set; or equivalently $X$ is quasi-compact.
\end{definition}

\begin{remark}
  If we are only considering schems (locally of) finite type over $\complex$, any morphism of ringed space over $\complex$ is automatically a morphism of {\em locally\/} ringed space over $\complex$.

\end{remark}

\begin{proposition}
  Let $\schemeOf{X}$ be a scheme locally of finite type over $\complex$, let $U \subseteq X$ be an open subset, then $(U, \mathcal{O}_{X}\mid_{U})$ is a scheme locally of finite type over $\complex$.
  \label{thm:restriction-of-scheme-of-finite-type-is-scheme-of-finite-type}
\end{proposition}
\begin{proof}
  Since $\schemeOf{X}$ is locally of finite type over $\complex$, it admits an affine open cover $U_{i}\cong\spec{R_{i}}$.
  Then $U = \bigcup_{i} U \cap U_{i}$; since for each $i$, $U \cap U_{i}$ is open in $U_{i} \cong \spec{R_{i}}$, it is a union of $\spec{{\left(R_{i}\right)}_{f_{j}}}$. This gives us an open covering by finitely generated $\complex$-algebra.
\end{proof}

\begin{proposition}
  If $\schemeOf{X}$ is a scheme locally of finite type over $\complex$, and $\spec A$ is an open affine subscheme of $\schemeOf{X}$, then $A$ is a finitely generated $\complex$-algebra as well.
\end{proposition}

\begin{proof}
  TBD %TODO: add proof here
\end{proof}

\begin{proposition}
  Let $\spec A$ and $\spec B$ be two affine finite schemes over $\complex$. Then any morphism $\Phi : \spec A \to \spec B$ is of the form $(\phi, \phi^{\star})$ where $\phi : B \to A$ is a $\complex$-algebra homomorphism.
\end{proposition}

\begin{proof}
  TBD %TODO: add proof here
\end{proof}

Maybe we should construct the following and recover the previous lemma as a corollary:
\begin{proposition}
  The category of affine scheme over $\complex$ is antequivalent to commutative $\complex$-algebras.
  The category of affine finite scheme over $\complex$ is antiequivalent to finitely generated commutative $\complex$-algebras.
  \[
    \asch_{\complex}^{\mathsf{op}}\cong \calgebra{\complex}
  \]
  \[
    \afsch{\complex}^{\mathsf{op}}\cong \fgcalgebra{\complex}
  \]
\end{proposition}

\begin{proposition}\label{thm:affine-scheme-closed-points-biject-algebra-hom}
  Let $\spec A$ be an affine finite scheme over $\complex$. We have that the set of closed points are in bijection to
  $\Hom_{\calgebra{\complex}}\left(A, \complex\right)$
\end{proposition}

\begin{proof}
  From~\cref{cor:maximal-ideal-algebra-hom}, we know that for each closed point $\mathfrak m$, i.e.\ a maximal ideal, there is a unique $\phi_{\mathfrak m} : A \to \complex$ whose kernel is $\mathfrak m$. Conversely, for any $\phi: A \to \complex$, $\ker \phi$ is certainly a prime ideal\footnote{$\ker \phi$ is equal to $(\specop \phi)(\star)$ where $\star$ is the unique point of $\specop\complex$}. Since $\phi$ is surjective\footnote{for each $c \in \complex$, $\phi(c \cdot 1) = c$}, its kernel is maximal.


  It remains to show that $\mathfrak m \mapsto \phi_{\mathfrak{m}}$ and $\phi \mapsto \ker \phi$ are inverse to each other. But this follows from the uniqueness from~\cref{cor:maximal-ideal-algebra-hom}:
  Let $\mathfrak{m}$ be a maximal ideal, then the $\ker \phi_{\mathfrak{m}}$ is exactly $\mathfrak m$ by definition of $\phi_{\mathfrak m}$;
  on the otherhand, if $\phi$ is an algebra homomorphism then $\phi$ and $\phi_{\ker \phi}$ are both algebra homomorphims that has kernel $\ker \phi$, hence must be equal.
\end{proof}

\begin{corollary}\label{cor:affine-closed-point-bijection-scheme-morphism}
  Let $\spec A$ be an affine finite scheme over $\complex$. We have that the set of closed points are in bijection to $\Hom_{\sch/\complex}{\left(\spec{\complex}, \spec A\right)}$
\end{corollary}

\begin{proposition}
  If $\schemeOf{X}$ is a scheme locally of finite type over $\complex$, then the set of closed points of $X$ is in bijection with $X(\complex) := \Hom_{\sch/\complex}\left(\spec{\complex}, \schemeOf{X}\right)$.
\end{proposition}


\begin{proof}
  Let $x \in X$ be a closed point and an affine open neighbourhood of $x \in U \cong \spec A$ where $A$ is a finite $\complex$-algebra. Thus the $x$ corresponds to a morphism $\Phi_{A}$ between $\spec \complex$ and $\spec A$ by \cref{cor:affine-closed-point-bijection-scheme-morphism}; we define $\Psi_{x}$ to be the composition of
  \begin{center}
    \begin{tikzcd}
      \spec\complex \ar[r, "\Phi_{A}"] & \spec A \ar[r, "\sim"]  & {\left(U, {\mathcal{O}_{X}\!\mid_{U}}\right)} \ar[r, hookrightarrow] & \schemeOf{X}.
    \end{tikzcd}
  \end{center}
  Moreover, $\Psi_{x}$ does not dependent on the choice of affine neighbourhood $\specop A$: support $x \in \specop A \cap \specop B$, then $\specop A \cap \specop B$ admits an open covering of spectra of finitely generated $\complex$-algebras by \Cref{thm:restriction-of-scheme-of-finite-type-is-scheme-of-finite-type}. Thus we can find a finitely generated $\complex$-algebra $C$ such that $\specop C \subseteq \specop A \cap \specop B$.
  \begin{center}
    \begin{tikzcd}
      & & \spec A \arrow[hookrightarrow]{rd} & \\
      \spec{\complex} \arrow{r}{\Phi_{C}} & \spec{C} \arrow[hookrightarrow]{ru} \arrow[hookrightarrow]{rd} & & \schemeOf{X}, \\
      & & \spec B \arrow[hookrightarrow]{ur} &
    \end{tikzcd}
  \end{center}
  where $ (\_: \specop C \hookrightarrow \specop A) \circ \Phi_{C}$ is exactly $\Phi_{A}$ and $(\_{: \specop C \hookrightarrow \specop B}) \circ \Phi_{C}$ is exactly $\Phi_{B}$ by \Cref{cor:affine-closed-point-bijection-scheme-morphism}; thus both composition in the commutative square above is $\Psi_{x}$, in another word, $\Psi_{x}$ is independent from the choice of affine neighbourhood.

  On the otherhand, if we are given a morphism $\Psi : \spec\complex \to \spec A$, let us denote $x$ to be the image of the unique point in $\specop \complex$ under $\Psi$; we want to show that $x$ is a closed point. Since affine open set forms a basis, we only need to check that, for any affine open $\spec A \hookrightarrow \schemeOf{X}$, $x$ is closed in $\specop A$. We consider the factorisation of $\Psi$:
  \begin{center}
    \begin{tikzcd}[column sep=large]
      \spec\complex \arrow{r}{\specop{\psi}} & {\spec A} \arrow[hookrightarrow]{r} & \schemeOf{X},
    \end{tikzcd}
  \end{center}
  where $\psi$ is a $\complex$-algebra homomorphism $A \to \complex$ such that $\specop{\psi} = \Psi|_{\specop A}$, hence by \Cref{cor:affine-closed-point-bijection-scheme-morphism}, we have $x$ is closed in $\spec A$.
  % FIXME: better explanation is required here.
  The two construction above is bijection is verified as the following:
  \begin{enumerate}
    \item Let $x$ be a closed point, then it corresponds to $\Psi_{x}$, but the image of the unique point in $\specop \complex$ under $\Psi_{x}$ is $x$;
    \item if $\Phi$ is a morphism $\spec\complex \to X$ and denote the unique image as $x$,
          $\Phi$ factors through affine open neighbourhood of $x$ hence it is $\Psi_{x}$ because $\Psi_{x}$ does not dependent on the choice of affine neightbourhood.
  \end{enumerate}
\end{proof}

\begin{proposition}
  Let $\Phi = (\phi, \phi^{*}) : \schemeOf{X} \to \schemeOf{Y}$ be a morphism of schemes locally of finite type over $\complex$, then $\phi$ maps closed points of $X$ to closed points of $Y$.
  \label{thm:morph-maps-closed-points-to-closed-points}
\end{proposition}

% TODO: insert proof here
\begin{proof}
  TBD
\end{proof}

%%% Local Variables:
%%% mode: latex
%%% TeX-master: "../content"
%%% End:
