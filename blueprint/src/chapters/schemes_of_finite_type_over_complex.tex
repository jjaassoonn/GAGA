\chapter{Schemes of finite type over $\complex$}

\section{Some general scheme theory}

Before we can define our main object of interest, {\em schemes of finite type over $\complex$}, we need to introduce some preliminary notions.

We already have the following in {\sf mathlib4} thanks to Andrew Yang:
\begin{definition}[Local property of ring homomorphisms]\label{def:ring-hom-prop-local}
  \lean{RingHom.PropertyIsLocal}
  Let $P$ be a property of ring homomorphisms:
%   \begin{itemize}
%     \item
    we say the property $P$ is local if
    \begin{enumerate}
      \item if $P$ holds for $\phi : A \to B$, then $P$ holds for $\phi_S: S^{-1}A \to \left\langle f(S)\right\rangle^{-1} B$ for any submonoid $S \subseteq A$.
      \item Let $\phi : A \to B$ be a ring homomorphism, if $P$ holds for $A \stackrel{\phi}{\to} B \to B_{f_i}$ for some $\{f_i\} \subseteq B$ such that $\langle f_i\rangle = B$, then $P$ holds for $\phi$.
      \item Let $\phi : A \to B$ and $\psi : B \to C$ be two ring homomorphisms, if $P$ holds for $\phi$ and $\psi$, then $P$ holds for $\psi \circ \phi$.
      \item $P$ holds for $A \to A_f$ for all $f \in A$.
    \end{enumerate}
%   \end{itemize}
\end{definition}

\begin{proposition}
  The property ``finite type'' of ring homomorphisms is local in the sense of \Cref{def:ring-hom-prop-local}.
  \lean{finiteType_ofLocalizationSpan}
\end{proposition}

\begin{definition}
  \lean{AlgebraicGeometry.targetAffineLocally}
  If $P$ is a property of ring homomorphisms then the property {\rm affine locally}~$P$ of scheme morphism $(\phi, \phi^*): \schemeOf{X} \to \schemeOf{Y}$ holds if and only if $P$ holds for all ring homomorphism $\Gamma(U, \mathcal{O}_X) \to \Gamma(V, \mathcal{O}_Y)$ for all affine subsets $U \subseteq X$ and $V \subseteq Y$ such that $\phi(U) \le V$.

\end{definition}

\begin{definition}[Morphisms locally of finite type {\cite[\href{https://stacks.math.columbia.edu/tag/01T0}{01T0}]{stacks-project}}]
  \lean{AlgebraicGeometry.LocallyOfFiniteType}
Let $\Phi : \schemeOf{X} \to \schemeOf{Y} := (\phi, \phi^*)$ be a morphism of schemes. We say
\begin{enumerate}
    \item $\Phi$ is locally of finite type if for any affine open $V \subseteq Y$ and affine open $U \subseteq X$ such that $\phi(U) \subseteq V$, we have the induced map $\Gamma(U, \mathcal{O}_X) \to \Gamma(V, \mathcal{O}_V)$ is a ring map of finite type. In another word, $\Phi$ is affine locally a ring homomorphism of finite type.
    \item $\Phi$ is of finite type if it is locally of finite type and $\phi$ is quasi-compact.
\end{enumerate}
\end{definition}

\begin{proposition}\label{thm:affine-locally-open-immersion}
  Let $\Phi : \schemeOf{X} \to \schemeOf{Y}$ be an open immersion between schemes, then $\Phi$ is locally of finite type.
  \lean{AlgebraicGeometry.locallyOfFiniteTypeOfIsOpenImmersion}
\end{proposition}

\begin{proposition}\label{thm:affine-locally-source-cover}
  \lean{AlgebraicGeometry.LocallyOfFiniteType.source_openCover_iff}
  Let $\Phi : \schemeOf{X} \to \schemeOf{Y}$ be a morphism of schemes and $\mathcal{U} = \{U_{i}\}$ be an open covering of $X$, then affine locally $\Phi$ is locally of finite type if and only if $\Phi|_{U_i}$ is a morphism locally of finite type.
\end{proposition}

\begin{proposition}\label{thm:affine-cover-locally-of-finite-type}
  \lean{AlgebraicGeometry.LocallyOfFiniteType.affine_openCover_iff}
  Let $\Phi : \schemeOf{X} \to \schemeOf{Y}$ be a morphism of schemes and $\mathcal{U}=\{U_{i}\}$ be an affine open covering of $Y$. Consider the pullback cover $\mathcal{V} = \{V_{i}\}$ of $X$, and if for each $i$, there is
  an affine open cover $\mathcal{W}_{i} = \{W_{i, j}\}$ for $V_{i} \subseteq X$. Then $\Phi$ is locally of finite type if and only if, for each $i$ and $j$, the ring map
  $\Gamma(W_{i, j}, \mathcal{O}_{X}) \to \Gamma(U_{i}, \mathcal{O}_{Y})$ is of finite type.
\end{proposition}

\begin{proposition}\label{thm:locally_of_finite_type_comp}
  \lean{AlgebraicGeometry.locallyOfFiniteTypeOfComp}
  Composition of morphisms locally of finite type is locally of finite type.
\end{proposition}

\section{Basic definitions and properties}

\begin{definition}[Schemes locally of finite type over $\complex$]\label{def:SchemeLocallyOfFiniteType}
  \lean{SchemeLocallyOfFiniteTypeOverComplex}
  A scheme $\schemeOf{X}$ is locally of finite type over $\complex$ if $\schemeOf{X}$ is a scheme over $\complex$ and the structure morphism $\schemeOf{X} \to \spec{\complex}$ is a morphism locally of finite type.
\end{definition}

\begin{definition}[Schemes of finite type over $\complex$]
  \lean{SchemeOfFiniteType}
  A scheme $\schemeOf{X}$ is of finite type over $\complex$ if $\schemeOf{X}$ is locally of finite type over $\complex$ and the structure morphism is quasicompact.

\end{definition}

Let us unpack the \Cref{def:SchemeLocallyOfFiniteType} a little:


\begin{definition}[Affine open covering of spectra of finitely generated $\complex$-algebras]
  An affine open covering of spectra of finitely generated $\complex$-algebra for a scheme $\schemeOf{X}$ over $\complex$ is the following data:
    \begin{enumerate}
        \item indexing set: $I$;
        \item a family of finitely generated algebras: $R : I \to \fgcalgebra{\complex}$;
        \item a family of open immersions: for each $i \in I$, $\iota_i: \spec{R_i} \to \schemeOf{X}$;
        \item covering: $c : X \to I$ such that for each $x \in X$, $c_x \in \range{\iota_i}$.
    \end{enumerate}
\end{definition}

\begin{lemma}
    A scheme $\schemeOf X$ is locally of finite type over $\complex$ if it is a scheme over $\complex$ and it admits an affine open cover of spectra of finitely generated $\complex$-algebras.
\end{lemma}

\begin{proof}
    This is unpacking \Cref{def:SchemeLocallyOfFiniteType}
\end{proof}

% \begin{definition}[Scheme of finite type over $\complex$]\label{def:SchemeOfFiniteType}
%   % \lean{SchemeOfFiniteType}
%   % \uses{SchemeLocallyOfFiniteType}
%   A scheme $\schemeOf{X}$ is of finite type over $\complex$ if it is a scheme locally of finite type over $\complex$
%   and has finite indexing set; or equivalently $X$ is quasi-compact.
% \end{definition}

\begin{proposition}
  \lean{SchemeLocallyOfFiniteTypeOverComplex.restrict}
  \leanok{}
  \uses{def:SchemeLocallyOfFiniteType}
  Let $\schemeOf{X}$ be a scheme locally of finite type over $\complex$, let $U \subseteq X$ be an open subset, then $(U, \mathcal{O}_{X}\mid_{U})$ is a scheme locally of finite type over $\complex$.
  \label{thm:restriction-of-scheme-of-finite-type-is-scheme-of-finite-type}
\end{proposition}
\begin{proof}
  By \Cref{thm:affine-locally-open-immersion} and \Cref{thm:locally_of_finite_type_comp}, open immersions are locally of finite type and composition of morphisms locally of finite type is again locally of finite type, so
  \[
    \left(U, \mathcal{O}_{X}|_{U}\right) \hookrightarrow \schemeOf{X} \to \spec\complex
  \]
  is a morphism locally of finite type as well.
\end{proof}

\begin{proposition}\label{thm:affine-open-is-finite-algebra}
  If $\schemeOf{X}$ is a scheme locally of finite type over $\complex$, and $U = \spec A$ is an open affine subscheme of $\schemeOf{X}$, then $A \cong \Gamma(U, \mathcal{O}_{X})$ is a finitely generated $\complex$-algebra as well.
  \lean{SchemeLocallyOfFiniteTypeOverComplex.inst_sections_finite}
  \leanok
  \uses{def:SchemeLocallyOfFiniteType,thm:affine-cover-locally-of-finite-type}
\end{proposition}

\begin{proof}
  Consider the only open affine cover $\{\specop \complex\}$ of $\specop \complex$, its pullback cover is $\{X_{\complex} := X \times_{\specop\complex} \specop\complex\}$ where $X_{\complex}$ can be covered by $U_{i}\times_{\specop \complex}\specop \complex$ where $U_{i}$ runs over the collection of all affine open sets.
  Then the conclusion follows from \Cref{thm:affine-cover-locally-of-finite-type}.

  % $X \times_{\specop \complex} \specop \complex \cong X$ has an affine open covering obtained by the collection of all affine open set. $\spec A$ is certain in the open cover, then the ring map $\complex \to A$ is
  % finite as well.
\end{proof}

\begin{lemma} If $U \subseteq \specop A$ is an open subset where $A$ is a finite $\complex$-algebra, then $U$ admits a finite covering of $D(f_{1}),\dots, D(f_{n})$ where $D(x)$ is the basic open around $x \in A$.
\end{lemma}
\begin{proof}
  Since $U$ is open, its complement $U^{\complement}$ is of the form $V(I)$ for some ideal $I$. Since $A$ is a finite $\complex$-algebra, $I$ is the span of $\{f_{1},\dots, f_{n}\}$ for some $f_{i} \in A$. Thus $V(I)$ is $\bigcap_{i}V(f_{i}\cdot R)$ hence $U$ is $\bigcup_{i}D(f_{i})$.
\end{proof}

\begin{lemma}
   If $\schemeOf{X}$ is a scheme of finite type over $\complex$ and $V \subseteq X$ is an open subset, then $V$ is quasi-compact.
\end{lemma}
\begin{proof}
  Since $\schemeOf{X}$ is of finite type, it has a {\em\/ finite} affine covering $\mathcal{U}_{i} = \{U_{1},\dots, U_{n}\}$.
  It is sufficient to show that every open cover of $U_{i} \cap V$ has a finite subcover\footnote{finite union of quasicompact set is again quasicompact}. In another word, we only need to show if $\schemeOf{X} \cong \spec A$ is affine and $V$ is an open subset of $X$, then $V$ is quasicompact. Since $V$ is a {\em finite\/} union of $D(f_{1}),\dots, D(f_{n})$ for some $f_{i}$'s in $A$, we only need to show that $D(f)$ is quasicompact. Since $D(f)$ is affine, it is quasicompact.
\end{proof}

\begin{corollary}[restriction of scheme of finite type over $\complex$]
  \lean{SchemeOfFiniteType.restrict}
  Let $\schemeOf{X}$ be a scheme of finite type over $\complex$ and $U \subseteq X$ be open, then the restriction $(U, \mathcal{O}_{X}|_{U})$ is a scheme of finite type over $\complex$ as well.
\end{corollary}

\begin{proposition}
  Let $\spec A$ and $\spec B$ be two affine finite schemes over $\complex$. Then any morphism $\Phi : \spec A \to \spec B$ is of the form $(\phi, \phi^{\star})$ where $\phi : B \to A$ is a $\complex$-algebra homomorphism.
\end{proposition}

\begin{proof}
  We have a ring homomorphism $\phi$, just need to check it commutes with algebra map.
\end{proof}

Maybe we should construct the following and recover the previous lemma as a corollary:
\begin{proposition}
  The category of affine scheme over $\complex$ is antiquivalent to commutative $\complex$-algebras.
  The category of affine finite scheme over $\complex$ is antiequivalent to finitely generated commutative $\complex$-algebras.
  \[
    \asch_{\complex}^{\mathsf{op}}\cong \calgebra{\complex}
  \]
  \[
    \afsch{\complex}^{\mathsf{op}}\cong \fgcalgebra{\complex}
  \]
\end{proposition}

\section{Closed points}
In this section, we focus on the subset set of closed points of a scheme locally of finite type over $\complex$, preparing for the complex topology. Let us fix some notation first: for an arbitrary scheme $\schemeOf{X}$, we denote $\maxop{X}$ to be the set of all closed points of $X$ and $\maxspecop R$ to be the set of all maximal ideals of a ring $R$. Note that $\maxspecop R$ is exactly $\maxop{\specop{R}}$ and we use both interchangeably.



\begin{proposition}\label{thm:affine-scheme-closed-points-biject-algebra-hom}
  Let $\spec A$ be an affine finite scheme over $\complex$. We have that the set of closed points $\maxspecop A$ are in bijection to
  $\Hom_{\calgebra{\complex}}\left(A, \complex\right)$
\end{proposition}

\begin{proof}
  From~\cref{cor:maximal-ideal-algebra-hom}, we know that for each closed point $\mathfrak m$, i.e.\ a maximal ideal, there is a unique $\phi_{\mathfrak m} : A \to \complex$ whose kernel is $\mathfrak m$. Conversely, for any $\phi: A \to \complex$, $\ker \phi$ is certainly a prime ideal\footnote{$\ker \phi$ is equal to $(\specop \phi)(\star)$ where $\star$ is the unique point of $\specop\complex$}. Since $\phi$ is surjective\footnote{for each $c \in \complex$, $\phi(c \cdot 1) = c$}, its kernel is maximal.


  It remains to show that $\mathfrak m \mapsto \phi_{\mathfrak{m}}$ and $\phi \mapsto \ker \phi$ are inverse to each other. But this follows from the uniqueness from~\cref{cor:maximal-ideal-algebra-hom}:
  Let $\mathfrak{m}$ be a maximal ideal, then the $\ker \phi_{\mathfrak{m}}$ is exactly $\mathfrak m$ by definition of $\phi_{\mathfrak m}$;
  on the other hand, if $\phi$ is an algebra homomorphism then $\phi$ and $\phi_{\ker \phi}$ are both algebra homomorphism that has kernel $\ker \phi$, hence must be equal.
\end{proof}

\begin{corollary}\label{cor:affine-closed-point-bijection-scheme-morphism}
  Let $\spec A$ be an affine finite scheme over $\complex$. We have that the $\maxspecop A$ is in bijection with $\Hom_{\sch/\complex}\left(\spec\complex, \spec A\right)$
\end{corollary}

\begin{proposition}
  If $\schemeOf{X}$ is a scheme locally of finite type over $\complex$, then $\maxop X$ is in bijection with $X(\complex) := \Hom_{\sch/\complex}\left(\spec{\complex}, \schemeOf{X}\right)$,
  such that every closed point $p$, is the image of $\star$ of a unique morphism $\Phi_{p}$; and for each morphism $\Phi : \spec\complex\to\schemeOf{X}$, $\Phi(\star)$ is closed in $X$ where $\star$ is the unique point of $\specop\complex$.
\end{proposition}

\begin{proof}
  Let $x \in X$ be a closed point and an affine open neighbourhood of $x \in U \cong \spec A$ where $A$ is a finite $\complex$-algebra. Thus the $x$ corresponds to a morphism $\Phi_{A}$ between $\spec \complex$ and $\spec A$ by \cref{cor:affine-closed-point-bijection-scheme-morphism}; we define $\Psi_{x}$ to be the composition of
  \begin{center}
    \begin{tikzcd}
      \spec\complex \ar[r, "\Phi_{A}"] & \spec A \ar[r, "\sim"]  & {\left(U, {\mathcal{O}_{X}\!\mid_{U}}\right)} \ar[r, hookrightarrow] & \schemeOf{X}.
    \end{tikzcd}
  \end{center}
  Moreover, $\Psi_{x}$ does not dependent on the choice of affine neighbourhood $\specop A$: suppose $x \in \specop A \cap \specop B$, then $\specop A \cap \specop B$ admits an open covering of spectra of finitely generated $\complex$-algebras by \Cref{thm:restriction-of-scheme-of-finite-type-is-scheme-of-finite-type}. Thus we can find a finitely generated $\complex$-algebra $C$ such that $\specop C \subseteq \specop A \cap \specop B$.
  \begin{center}
    \begin{tikzcd}
      & & \spec A \arrow[hookrightarrow]{rd} & \\
      \spec{\complex} \arrow{r}{\Phi_{C}} & \spec{C} \arrow[hookrightarrow]{ru} \arrow[hookrightarrow]{rd} & & \schemeOf{X}, \\
      & & \spec B \arrow[hookrightarrow]{ur} &
    \end{tikzcd}
  \end{center}
  where $ (\_: \specop C \hookrightarrow \specop A) \circ \Phi_{C}$ is exactly $\Phi_{A}$ and $(\_{: \specop C \hookrightarrow \specop B}) \circ \Phi_{C}$ is exactly $\Phi_{B}$ by \Cref{cor:affine-closed-point-bijection-scheme-morphism}; thus both composition in the commutative square above is $\Psi_{x}$, in another word, $\Psi_{x}$ is independent from the choice of affine neighbourhood.

  On the otherhand, if we are given a morphism $\Psi : \spec\complex \to \spec A$, let us denote $x$ to be the image of the unique point in $\specop \complex$ under $\Psi$; we want to show that $x$ is a closed point. Since affine open set forms a basis, we only need to check that, for any affine open $\spec A \hookrightarrow \schemeOf{X}$, $x$ is closed in $\specop A$. We consider the factorisation of $\Psi$:
  \begin{center}
    \begin{tikzcd}[column sep=large]
      \spec\complex \arrow{r}{\specop{\psi}} & {\spec A} \arrow[hookrightarrow]{r} & \schemeOf{X},
    \end{tikzcd}
  \end{center}
  where $\psi$ is a $\complex$-algebra homomorphism $A \to \complex$ such that $\specop{\psi} = \Psi|_{\specop A}$, hence by \Cref{cor:affine-closed-point-bijection-scheme-morphism}, we have $x$ is closed in $\spec A$.
  % FIXME: better explanation is required here.
  The two construction above is bijection is verified as the following:
  \begin{enumerate}
    \item Let $x$ be a closed point, then it corresponds to $\Psi_{x}$, but the image of the unique point in $\specop \complex$ under $\Psi_{x}$ is $x$;
    \item if $\Phi$ is a morphism $\spec\complex \to X$ and denote the unique image as $x$,
          $\Phi$ factors through affine open neighbourhood of $x$ hence it is $\Psi_{x}$ because $\Psi_{x}$ does not dependent on the choice of affine neightbourhood.
  \end{enumerate}
\end{proof}

\begin{proposition}
  Let $\Phi = (\phi, \phi^{*}) : \schemeOf{X} \to \schemeOf{Y}$ be a morphism of schemes locally of finite type over $\complex$, then $\phi$ maps closed points of $X$ to closed points of $Y$. Thus, we have a well defined map $\maxop\phi : \maxop{X} \to \maxop{Y}$
  \label{thm:morph-maps-closed-points-to-closed-points}
\end{proposition}

\begin{proof}
  Let $x$ be a closed point in $X$, then $x$ corresponds to a unique $\Psi_{x} = (\psi_{x}, \psi_{x}^{*}) : \spec{\complex} \to X$ such that $\psi_{x}(\star) = x$ where $\star$ is the unique point of $\specop\complex$. The composite $\Phi \circ \Psi_{x}$ is a morphism $\spec\complex\to Y$ thus $\Phi \circ \Psi_{x}(\star)$ is closed in $Y$, since $\Phi \circ \Psi_{x}(\star) = \Phi(\Psi_{x}(\star)) = \Phi(x)$, we conclude that $\Phi(x)$ is a closed point in $Y$.
\end{proof}

When we talk about topology on $\maxop{X}$, we mean the subspace topology induced by the Zariski topology\footnote{Hopefully, I will be able to main the notation clearly: $\maxop X$ is Zariski and $\an{X}$ is analytic}.

\begin{corollary}
  Let $\Phi: \schemeOf{X} \to \schemeOf{Y}$ be an open immersion, then $\maxop\phi$ is an embedding.
\end{corollary}

\begin{proof}
  Write $\Phi= (\phi, \phi^{*})$, note that $\phi$ is necessarily an embedding $X \hookrightarrow Y$ thus $\maxop\phi$ being the restriction of $\phi$ must be embedding as well.
\end{proof}

\begin{remark}
  Let $X$ be a scheme locally of finite type over $\complex$ and $\mathcal{U} = \{U_{i}\}$ be an open cover of $X$. Then $\maxop{X} = \bigcup_{i}\maxop{U_{i}}$, so that $\maxop{\mathcal{U}} = \{\maxop{U_{i}}\}$ is a Zariski open cover for $\maxop{X}$
\end{remark}

\begin{remark}
  If we are only considering schemes (locally of) finite type over $\complex$, any morphism of ringed space over $\complex$ is automatically a morphism of {\em locally\/} ringed space over $\complex$.
\end{remark}

%%% Local Variables:
%%% mode: latex
%%% TeX-master: "../content"
%%% End:
